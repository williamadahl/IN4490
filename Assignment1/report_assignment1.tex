\documentclass[a4paper, norsk, 12pt]{article}
\usepackage{babel,textcomp}
\usepackage[norsk]{isodate}
\usepackage[utf8]{inputenc}
\usepackage[T1]{fontenc}
\usepackage{listingsutf8}
\usepackage{graphicx}
\usepackage{amsmath,scalerel}
\usepackage{fancyhdr}
\usepackage{enumitem}
\usepackage{paralist}
\usepackage{dsfont}
\usepackage{upgreek}
\usepackage[usenames, dvipsnames]{color}


\definecolor{dkgreen}{rgb}{0,0.6,0}
\definecolor{gray}{rgb}{0.5,0.5,0.5}
\definecolor{mauve}{rgb}{0.58,0,0.82}


\lstset{frame=tb,
  language=matlab,
  aboveskip=3mm,
  belowskip=3mm,
  showstringspaces=false,
  columns=flexible,
  basicstyle={\small\ttfamily},
  numbers=none,
  numberstyle=\tiny\color{gray},
  keywordstyle=\color{blue},
  commentstyle=\color{dkgreen},
  stringstyle=\color{mauve},
  breaklines=true,
  breakatwhitespace=true,
  tabsize=3,
  inputencoding=ansinew,
  extendedchars=true,
  literate={æ}{{\ae}}1 {å}{{\aa}}1 {ø}{{\o}}1 {Æ}{{\AE}}1 {Å}{{\AA}}1 {Ø}{{\O}}1,
}

\pagestyle{fancy}
\fancyhf{}
\rhead{21.09.2018}
\chead{IN4490 REPORT}
\lhead{William Arild Dahl}
\rfoot{\thepage}

\begin{document}
\begin{titlepage}

\newcommand{\HRule}{\rule{\linewidth}{0.5mm}} % Defines a new command for the horizontal lines, change thickness here

\center % Center everything on the page

%----------------------------------------------------------------------------------------
%	HEADING SECTIONS
%----------------------------------------------------------------------------------------

\textsc{\LARGE Universitetet i Oslo}\\[1.5cm] % Name of your university/college
\textsc{\Large IN4490}\\[0.5cm] % Major heading such as course name
\textsc{\large Biologically inspired computing}\\[0.5cm] % Minor heading such as course title

%----------------------------------------------------------------------------------------
%	TITLE SECTION
%----------------------------------------------------------------------------------------

\HRule \\[0.4cm]
{ \huge \bfseries Traveling Salesman Problem}\\[0.4cm] % Title of your document
\HRule \\[1.5cm]


%----------------------------------------------------------------------------------------
%	AUTHOR SECTION
%----------------------------------------------------------------------------------------

%\begin{minipage}{0.4\textwidth}
%\begin{flushleft} \large
%\emph{Author:}\\
%John \textsc{Smith} % Your name
%\end{flushleft}
%\end{minipage}
%~
%\begin{minipage}{0.4\textwidth}
%\begin{flushright} \large
%\emph{Supervisor:} \\
%Dr. James \textsc{Smith} % Supervisor's Name
%\end{flushright}
%\end{minipage}\\[2cm]

% If you don't want a supervisor, uncomment the two lines below and remove the section above
\Large

William Arild Dahl\\ % Your name
\textsc{Username:} williada\\[3cm]
Remember to add info about discrete optimization


%----------------------------------------------------------------------------------------
%	DATE SECTION
%----------------------------------------------------------------------------------------

{\large 21.09.2018}\\[2cm] % Date, change the \today to a set date if you want to be precise

%----------------------------------------------------------------------------------------
%	LOGO SECTION
%----------------------------------------------------------------------------------------

\includegraphics[width=1\textwidth]{uio_banner.png}\\[1cm] % Include a department/university logo - this will require the graphicx package

%----------------------------------------------------------------------------------------

\vfill % Fill the rest of the page with whitespace
\end{titlepage}

%------------------------------------------------------------------------------
%               Text section
%------------------------------------------------------------------------------

\section{Exhaustive Search}
Brute force a very simple approach to the TSP, and in my case, the only solution i knew about before starting this course. It explores every possible tour and is therefore guaranteed to find the optimal solution. Exhaustive search, also called Brute Force has a \textbf{very} high complexity: $\mathcal{O}{n!}$ with $n$ being the number of cities. 
\newline \newline 
Running exhaustive search with 6-10 cities on a Intel Core m5-6Y54 CPU @ 1.10GHz, produced the following results: 
	\begin{lstlisting}
	Runtime for 6 cities is: 0.00172 seconds.
	Minimum distance is: 5018.81.
	
	Runtime for 7 cities is: 0.01516 seconds.
	Minimum distance is: 5487.89.
		
	Runtime for 8 cities is: 0.13097 seconds.
	Minimum distance is: 6667.49.
	
	Runtime for 9 cities is: 1.23250 seconds.
	Minimum distance is: 6678.55.

	Runtime for 10 cities is: 14.47876 seconds.
	Minimum distance is: 7486.31.

	Shortest tour for 10 cities:
	['Copenhagen', 'Hamburg', 'Brussels', 'Dublin', 'Barcelona', 'Belgrade', 'Istanbul', 'Bucharest', 'Budapest', 'Berlin']
	\end{lstlisting}
As we can see, the runtime increases dramatically with each added city to the problem, this is due to the factorial complexity of the algorithm.I am estimating that my CPU can check somewhere in the range of 250,000 - 280,000 permutations/second. 
\[ 
	t =  \frac{n!}{p}
\]
Where $t$ is time in seconds , $n$ is the number of cities, and $p$ is the number of permutations/second. 
If I were to run my code with 24 cities we are looking at a runtime of about $4.4\times 10^{12}$ years\footnote{318 times longer than the universe has existed.}.

\section{Hill Climbing}
Hill Climbing is a stochastic method. I solved it by choosing a random starting point(permutation) and calculating its tour length. I then swap two random genes in the genome and calculate this genomes tour length. I repeat this random swapping 1000 times and always keep the better result. This way the algorithm is 'climbing' up towards a local optima, or preferably a global optima if I am lucky. Hill Climbing is greedy, so it will only and will always pick the better neighbor. That means that there is a great possibility that the algorithm will end up in a local optima. I choose to use 20 random starting points to ensure more exploring.
\newline\newline
Here are the results of 20 random starting points and 1000 gene swaps: 
	\begin{lstlisting}
	Runtime for 10 cities visited: 0.1891 seconds.
	Length of best tour is: 7486.31.
 	Length of worst tour is: 8377.24.
 	Length of average tour is: 7652.89.
 	Standard deviation is: 261.32. 
 	Route of best tour is:['Copenhagen', 'Hamburg', 'Brussels', 'Dublin', 'Barcelona', 'Belgrade', 'Istanbul', 'Bucharest', 'Budapest', 'Berlin'].

	Runtime for 24 cities visited: 0.2843 seconds. 
	Length of best tour is: 12967.29.
 	Length of worst tour is: 17152.05.
 	Length of average tour is: 14917.95.
 	Standard deviation is: 1183.94. 
 	Route of best tour is:['London', 'Paris', 'Brussels', 'Prague', 'Vienna', 'Budapest', 'Bucharest', 'Istanbul', 'Sofia', 'Belgrade', 'Warsaw', 'Kiev', 'Moscow', 'Saint Petersburg', 'Stockholm', 'Copenhagen', 'Hamburg', 'Berlin', 'Munich', 'Milan', 'Rome', 'Barcelona', 'Madrid', 'Dublin'].
	\end{lstlisting}
	As we can see from the algorithm is significantly faster than exhaustive search. In this run for $N = 10$, hill climbing was able to find the global optima, which is confirmed by exhaustive search. However, this is not always the case for every run, but the algorithm gets fairy close every time. We can also observe that the increase to $N = 24$ did not impact the runtime significantly. I am fairly sure that the tour for 24 cities is \textbf{not} the global optima, but considering the runtime of an exhaustive search, the answer is probably good enough.   

\end{document}